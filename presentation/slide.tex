%%%%%%%%%%%%%%%%%%%%%%%%%%%%%%%%%%%%%%%%%%%%%%%%%%%%%%
% A Beamer template for University of Wollongong     %
% Based on THU beamer theme                          %
% Author: Qiuyu Lu                                   %
% Date: July 2024                                    %
% LPPL Licensed.                                     %
%%%%%%%%%%%%%%%%%%%%%%%%%%%%%%%%%%%%%%%%%%%%%%%%%%%%%%

\documentclass[serif, aspectratio=169]{beamer}
%\documentclass[serif]{beamer}  % for 4:3 ratio
\usepackage[T1]{fontenc} 
\usepackage{fourier} % see "http://faq.ktug.org/wiki/uploads/MathFonts.pdf" for other options
\usepackage{hyperref}
\usepackage{latexsym,amsmath,xcolor,multicol,booktabs,calligra}
\usepackage{graphicx,pstricks,listings,stackengine}
\usepackage{lipsum}

\author[Danko, Strygin]{Danila Danko \inst{1} \and Nikita Strygin \inst{1} \and \newline \newline {Supervisor: Nikolai Kudasov \inst{1}}}
\title{Query-Driven Language Server Architecture using Second-Order Abstract Syntax}

\institute{
    \inst{1}Innopolis University
}
\date{\small \today}

\usepackage{UoWstyle}

% defs
\def\cmd#1{\texttt{\color{red}\footnotesize $\backslash$#1}}
\def\env#1{\texttt{\color{blue}\footnotesize #1}}
\definecolor{deepblue}{rgb}{0,0,0.5}
\definecolor{deepred}{RGB}{153,0,0}
\definecolor{deepgreen}{rgb}{0,0.5,0}
\definecolor{halfgray}{gray}{0.55}

\lstset{
    basicstyle=\ttfamily\small,
    keywordstyle=\bfseries\color{deepblue},
    emphstyle=\ttfamily\color{deepred},    % Custom highlighting style
    stringstyle=\color{deepgreen},
    numbers=left,
    numberstyle=\small\color{halfgray},
    rulesepcolor=\color{red!20!green!20!blue!20},
    frame=shadowbox,
}


\begin{document}

\begin{frame}
    \titlepage
\end{frame}

\begin{frame}
    \tableofcontents[sectionstyle=show,
        subsectionstyle=show/shaded/hide,
        subsubsectionstyle=show/shaded/hide]
\end{frame}

\section{Introduction}

\begin{frame}{Problem}
    \begin{itemize}
        % [<+-| alert+>] % stepwise alerts
        \item Programming language researchers, enthusiasts, and students sometimes prototype new languages.
        \item Some of the language authors try to make their languages user-friendly by supporting the language server protocol (LSP) \cite{noauthor_language_server_protocol_2024}.
        \item Language servers for all mainstream languages support "go to definition", "lookup the type on hover", "rename all occurences".
        \item There exist frameworks that support building languages integrated with the LSP, e.g., Langium \cite{noauthor_langium_nodate} for languages implemented in \texttt{TypeScript} and \texttt{lsp-tree-sitter} \cite{noauthor_neomuttlsp-tree-sitter_2024} for languages implemented in Python.
        \item However, to our best knowledge, there is no such framework for languages implemented in Haskell.
    \end{itemize}
\end{frame}

\begin{frame}{Solution}
    \begin{itemize}
        % [<+-| alert+>] % stepwise alerts
        \item Language servers share some features, such as "go to definition".
        \item Our assumption is that some of these features can be implemented at least partially in a language-agnostic way.
        \item A promising approach is to work with the Second-Order Abstract Syntax (SOAS) of a language.
        \item SOAS allows to work with scope resolution ("go to definition"), variable bindings ("lookup the type on hover"), and substitution ("rename all occurences") in a language-agnostic way.
        \item We plan to use the Free Foil \cite{kudasov_free_2024} to generate SOAS of a language.
    \end{itemize}
\end{frame}

\section{Future Work}

\subsection{Literature review}

\begin{frame}{Literature review}
    \begin{itemize}
        \item Read \cite{maclaurin_foil_2022} and \cite{kudasov_free_2024}
              % \item \lipsum[2][1-4]
              % \item \lipsum[2][5-9]
              % \item Results accessible at \newline \url{https://scholar.google.com}
    \end{itemize}
\end{frame}

\subsection{Practice with Free Foil}

\begin{frame}{Practice with Free Foil}
    \begin{itemize}
        \item Complete exercises with ASTs
        \item Complete them with Free Foil
    \end{itemize}
\end{frame}

\subsection{Practice with a simple typed language}

\begin{frame}{Practice with a simple typed language}
    \begin{itemize}
        \item Implement a type checker and interpreter for a simple typed language, e.g. Simply Typed Lambda Calculus.
        \item Features:
              \begin{itemize}
                  \item Single module on the input and the output
                  \item Use BNFC for pretty-printing and parsing
                  \item Use Free Foil and Template Haskell
                  \item Show error location in the code
              \end{itemize}
    \end{itemize}

\end{frame}

% \section{Literature Review}

% \subsection{GPT3-derived Models DALLE \& CLIP}

% \begin{frame}
%     \begin{itemize}
%         \item \lipsum[2][1-4]
%         \item \lipsum[2][5-9]
%         \item Results accessible at \newline \url{https://scholar.google.com}
%     \end{itemize}
% \end{frame}


% \section{Methods}

% \subsection{Diffusion Model}

% \begin{frame}{Title}
%     \begin{itemize}
%         \item \lipsum[3][1-4]
%     \end{itemize}
%     \begin{table}[h]
%         \centering
%         \begin{tabular}{c|c}
%             Microsoft\textsuperscript{\textregistered}  Windows & Apple\textsuperscript{\textregistered}  Mac OS \\
%             \hline
%             Windows-Kernel & Unix-like \\
%             Arm, Intel & Intel, Apple Silicon \\
%             Sudden update & Stable update \\
%             Less security & More security \\
%             ... & ... \\
%         \end{tabular}
%     \end{table}
% \end{frame}

% \begin{frame}{Algorithms}
%     \begin{exampleblock}{Non-Numbering Formula}
%         \begin{equation*}
%             J(\theta) = \mathbb{E}_{\pi_\theta}[G_t] = \sum_{s\in\mathcal{S}} d^\pi (s)V^\pi(s)=\sum_{s\in\mathcal{S}} d^\pi(s)\sum_{a\in\mathcal{A}}\pi_\theta(a|s)Q^\pi(s,a)
%         \end{equation*}
%     \end{exampleblock}
%     \begin{exampleblock}{Multi-Row Formula\footnote{If text appears in the formula, use $\backslash$mathrm\{\} or $\backslash$text\{\} instead}}
%         \begin{align}
%             Q_\mathrm{target}&=r+\gamma Q^\pi(s^\prime, \pi_\theta(s^\prime)+\epsilon)\\
%             \epsilon&\sim\mathrm{clip}(\mathcal{N}(0, \sigma), -c, c)\nonumber
%         \end{align}
%     \end{exampleblock}
% \end{frame}

% \begin{frame}
%     \begin{exampleblock}{Numbered Multi-line Formula}
%         % Taken from Mathmode.tex
%         \begin{multline}
%             A=\lim_{n\rightarrow\infty}\Delta x\left(a^{2}+\left(a^{2}+2a\Delta x+\left(\Delta x\right)^{2}\right)\right.\label{eq:reset}\\
%             +\left(a^{2}+2\cdot2a\Delta x+2^{2}\left(\Delta x\right)^{2}\right)\\
%             +\left(a^{2}+2\cdot3a\Delta x+3^{2}\left(\Delta x\right)^{2}\right)\\
%             +\ldots\\
%             \left.+\left(a^{2}+2\cdot(n-1)a\Delta x+(n-1)^{2}\left(\Delta x\right)^{2}\right)\right)\\
%             =\frac{1}{3}\left(b^{3}-a^{3}\right)
%         \end{multline}
%     \end{exampleblock}
% \end{frame}

% \begin{frame}{Graphics and Columns}
%     \begin{minipage}[c]{0.3\linewidth}
%         \psset{unit=0.8cm}
%         \begin{pspicture}(-1.75,-3)(3.25,4)
%             \psline[linewidth=0.25pt](0,0)(0,4)
%             \rput[tl]{0}(0.2,2){$\vec e_z$}
%             \rput[tr]{0}(-0.9,1.4){$\vec e$}
%             \rput[tl]{0}(2.8,-1.1){$\vec C_{ptm{ext}}$}
%             \rput[br]{0}(-0.3,2.1){$\theta$}
%             \rput{25}(0,0){%
%             \psframe[fillstyle=solid,fillcolor=lightgray,linewidth=.8pt](-0.1,-3.2)(0.1,0)}
%             \rput{25}(0,0){%
%             \psellipse[fillstyle=solid,fillcolor=yellow,linewidth=3pt](0,0)(1.5,0.5)}
%             \rput{25}(0,0){%
%             \psframe[fillstyle=solid,fillcolor=lightgray,linewidth=.8pt](-0.1,0)(0.1,3.2)}
%             \rput{25}(0,0){\psline[linecolor=red,linewidth=1.5pt]{->}(0,0)(0.,2)}
% %           \psRotation{0}(0,3.5){$\dot\phi$}
% %           \psRotation{25}(-1.2,2.6){$\dot\psi$}
%             \psline[linecolor=red,linewidth=1.25pt]{->}(0,0)(0,2)
%             \psline[linecolor=red,linewidth=1.25pt]{->}(0,0)(3,-1)
%             \psline[linecolor=red,linewidth=1.25pt]{->}(0,0)(2.85,-0.95)
%             \psarc{->}{2.1}{90}{112.5}
%             \rput[bl](.1,.01){C}
%         \end{pspicture}
%     \end{minipage}\hspace{2cm}
%     \begin{minipage}{0.5\linewidth}
%         \medskip
%         % \hspace{2cm}
%         \begin{figure}[h]
%             \centering
%             \includegraphics[height=.4\textheight]{pic/sample.pdf}
%         \end{figure}
%     \end{minipage}
% \end{frame}

% \begin{frame}[fragile]{\LaTeX{} Common Commands}
%     \begin{exampleblock}{Commands}
%         \centering
%         \footnotesize
%         \begin{tabular}{llll}
%             \cmd{chapter} & \cmd{section} & \cmd{subsection} & \cmd{paragraph} \\
%             chapter & section & sub-section & paragraph \\\hline
%             \cmd{centering} & \cmd{emph} & \cmd{verb} & \cmd{url} \\
%             center & emphasize & original & hyperlink \\\hline
%             \cmd{footnote} & \cmd{item} & \cmd{caption} & \cmd{includegraphics} \\
%             footnote & list item & caption & insert image \\\hline
%             \cmd{label} & \cmd{cite} & \cmd{ref} \\
%             label & citation & refer\\\hline
%         \end{tabular}
%     \end{exampleblock}
%     \begin{exampleblock}{Environment}
%         \centering
%         \footnotesize
%         \begin{tabular}{lll}
%             \env{table} & \env{figure} & \env{equation}\\
%             table & figure & formula \\\hline
%             \env{itemize} & \env{enumerate} & \env{description}\\
%             non-numbering item & numbering item & description \\\hline
%         \end{tabular}
%     \end{exampleblock}
% \end{frame}

% \begin{frame}[fragile]{\LaTeX{} Examples of environmental commands}
%     \begin{minipage}{0.5\linewidth}
% \begin{lstlisting}[language=TeX]
% \begin{itemize}
%   \item A \item B
%   \item C
%   \begin{itemize}
%     \item C-1
%   \end{itemize}
% \end{itemize}
% \end{lstlisting}
%     \end{minipage}\hspace{1cm}
%     \begin{minipage}{0.3\linewidth}
%         \begin{itemize}
%             \item A
%             \item B
%             \item C
%             \begin{itemize}
%                 \item C-1
%             \end{itemize}
%         \end{itemize}
%     \end{minipage}
%     \medskip
%     \pause
%     \begin{minipage}{0.5\linewidth}
% \begin{lstlisting}[language=TeX]
% \begin{enumerate}
%   \item A \item B
%   \item C
%   \begin{itemize}
%     \item[n+e]
%   \end{itemize}
% \end{enumerate}
% \end{lstlisting}
%     \end{minipage}\hspace{1cm}
%     \begin{minipage}{0.3\linewidth}
%         \begin{enumerate}
%             \item A
%             \item B
%             \item C
%             \begin{itemize}
%                 \item[n+e]
%             \end{itemize}
%         \end{enumerate}
%     \end{minipage}
% \end{frame}

% \begin{frame}[fragile]{\LaTeX{} Formulas}
%     \begin{columns}
%         \begin{column}{.55\textwidth}
% \begin{lstlisting}[language=TeX]
% $V = \frac{4}{3}\pi r^3$

% \[
%   V = \frac{4}{3}\pi r^3
% \]

% \begin{equation}
%   \label{eq:vsphere}
%   V = \frac{4}{3}\pi r^3
% \end{equation}
% \end{lstlisting}
%         \end{column}
%         \begin{column}{.4\textwidth}
%             $V = \frac{4}{3}\pi r^3$
%             \[
%                 V = \frac{4}{3}\pi r^3
%             \]
%             \begin{equation}
%                 \label{eq:vsphere}
%                 V = \frac{4}{3}\pi r^3
%             \end{equation}
%         \end{column}
%     \end{columns}
%     \begin{itemize}
%         \item more information \href{https://ja.overleaf.com/learn/latex/Mathematical_expressions}{\color{purple}{here}}
%     \end{itemize}
% \end{frame}

% \begin{frame}[fragile]
%     \begin{columns}
%         \column{.6\textwidth}
% \begin{lstlisting}[language=TeX]
% \begin{table}[htbp]
%   \caption{numbers & meaning}
%   \label{tab:number}
%   \centering
%   \begin{tabular}{cl}
%     \toprule
%     number & meaning \\
%     \midrule
%     1 & 4.0 \\
%     2 & 3.7 \\
%     \bottomrule
%   \end{tabular}
% \end{table}
% \end{lstlisting}
%         \column{.4\textwidth}
%         \begin{table}[htpb]
%             \centering
%             \caption{numbers \& meaning}
%             \label{tab:number}
%             \begin{tabular}{cl}\toprule
%                 numbers & meaning \\\midrule
%                 1 & 4.0\\
%                 2 & 3.7\\\bottomrule
%             \end{tabular}
%         \end{table}
%         \normalsize formula~(\ref{eq:vsphere}) at previous slide and Table~\ref{tab:number}.
%     \end{columns}
% \end{frame}

% \section{Results}
% \begin{frame}
%     \begin{itemize}
%         \item \lipsum[4][1-4]
%         \item \lipsum[4][5-9]
%         \item \lipsum[5][1-4]
%         \item \lipsum[5][5-8]
%     \end{itemize}
% \end{frame}

\section{References}

\begin{frame}[allowframebreaks]
    \bibliography{ref}
    \bibliographystyle{ieeetr}
    % \nocite{*} % used here because no citation happens in slides
    % if there are too many try use:
    \tiny\bibliographystyle{alpha}
\end{frame}


\begin{frame}
    \begin{center}
        {\Huge Thank You}
    \end{center}
\end{frame}

\end{document}