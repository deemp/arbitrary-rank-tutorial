\chapter{Introduction}
\label{chap:Introduction}

\epigraph{Any sufficiently advanced technology is indistinguishable from magic.}{\textit{Arthur C. Clarke}}

For me, the Glasgow Haskell Compiler (GHC) \cite{ghc-site-2025} represented that "sufficiently advanced technology."
Despite programming in Haskell for several years prior to beginning this thesis, I had not yet explored the internal workings of its primary compiler.

% TODO: Consider adding specific links to key resources like the GHC wiki, relevant publications, or community forums?
Fortunately, GHC is open source, its implementation is grounded in scientific publications, and a wealth of resources --- including the project wiki, online presentations, and supportive online communities with compiler developers and experts --- facilitated my learning.

To deepen my understanding of GHC's internals and create a practical context for further Haskell programming, I decided to implement a small, extensible, typed functional programming language employing approaches similar to those used in GHC. I called this language Practical Lambda Calculus (PLC) as it was based on a variant of lambda calculus introduced in the paper \citetitle{jones-practical-2007} \cite{jones-practical-2007}.

The project aimed to produce a complete toolchain for my language, including a parser, a type inference engine, an interpreter, a language server, and a corresponding VS Code extension.

My search on GitHub revealed a lack of well-documented, open-source implementations of simple languages that feature parametric predicative higher-rank polymorphism, include the aforementioned components, and mirror GHC's architecture. This work attempts to partially address this gap by documenting the design options considered and detailing my final implementation. The implementation \cite{deemp-higher-rank-free-foil} is available on GitHub under the MIT license.

\section{Overview}

This thesis is structured as follows:

\Cref{chap:LiteratureReview} reviews several abstract syntax tree (AST) representations and type inference algorithms, including those ultimately chosen for my implementation.
Subsequently, \Cref{chap:DesignImplementation} details the theoretical foundations and implementation specifics of my language.
\Cref{chap:EvaluationDiscussion} evaluates the results obtained and discusses the development process and experience.
Following this, \Cref{chap:Conclusion} outlines potential directions for future work.
Finally, \Cref{chap:Appendix} contains relevant code snippets referenced throughout the thesis.