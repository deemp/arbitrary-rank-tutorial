\chapter{Introduction}
\label{chap:Introduction}

\epigraph{Any sufficiently advanced technology is indistinguishable from magic.}{\textit{Arthur C. Clarke}}

% Section 1: Background and Motivation - Refined to set the stage more precisely.
\section{Background and Motivation}

Advanced type systems are a cornerstone of modern functional programming, enabling developers to write safer, more expressive, and more maintainable code. One such powerful feature is \textbf{arbitrary-rank polymorphism}, which allows functions to accept other polymorphic functions as arguments. This capability, while seemingly abstract, is fundamental to implementing advanced language features like GADTs, type-level programming, and sophisticated forms of generic programming.

The theoretical foundations for implementing this feature in a practical compiler were laid out in the seminal paper "Practical Type Inference for Arbitrary-Rank Polymorphism" by Peyton Jones et al. \cite{jones-practical-2007}. The Glasgow Haskell Compiler (GHC) \cite{ghc-site-2025}, the de facto standard compiler for the Haskell programming language, serves as the most prominent real-world implementation of these ideas. For many developers and language enthusiasts, including the author, the inner workings of GHC represent a pinnacle of compiler technology --- a "sufficiently advanced technology" that can feel like magic.

% Section 2: The Problem - Made more specific, referencing the key technical challenges.
\section[The Problem: A Pedagogical Gap in Understanding Compiler Internals]{The Problem: A Pedagogical Gap \\ in Understanding Compiler Internals}

Despite the existence of both foundational theory and a production-grade implementation, a significant pedagogical gap remains. Aspiring language developers and students face a steep learning curve when trying to understand how the elegant theory of arbitrary-rank polymorphism translates into practical code. This gap is not merely one of complexity, but of translation between different models of computation:
\begin{itemize}
    \item The academic literature, including \cite{jones-practical-2007}, presents a theoretically dense system built on the intricate dance of \textbf{subsumption}, \textbf{deep skolemization}, and \textbf{bidirectional checking}. While formally sound \cite{practical-type-inference-proofs}, the paper's description of an \textbf{eager unification} algorithm omits many of the practical engineering details required to build a robust system.
    \item The GHC source code, while an invaluable resource, is a large, highly-optimized production system. Its modern counterpart MicroHs \cite{augustsson-microhs-2024} \cite{augustss-microhs-2025} is large too. Their current \textbf{constraint-based} type inference architecture is a significant evolution from the eager model described in the foundational papers, making it difficult for a newcomer to trace the connection between theory and implementation.
    \item Existing educational compilers for Haskell like Hugs \cite{hugs-haskell} are outdated or do not incorporate these modern architectural patterns, leaving students without a clear bridge from first principles to the state of the art.
\end{itemize}
There is a clear need for a resource that bridges this gap --- a "middle ground" that is more concrete than a paper but more focused and accessible than a full production compiler, one that explicitly demonstrates the architectural evolution from eager to constraint-based inference.

% Section 3: Contribution - Reframed to be a direct answer to the refined problem.
\section{Contribution: The \Arralac Compiler}

% 10.5555/1177220

To address this pedagogical gap, this thesis presents the design and implementation of a small, typed functional language named \Arralac (\textbf{Ar}bitrary-\textbf{ra}nk polymorphism + \textbf{la}mbda \textbf{c}alculus). The \Arralac compiler (in the sense that it translates a source code into a target program \cite{10.5555/1177220}) serves as a well-documented, tutorial implementation that explicitly models the architectural principles of a modern typechecker for arbitrary-rank polymorphism.

The central thesis of this work is that a focused, modern, and interactive implementation that consciously diverges from older algorithmic models can serve as a more effective learning tool than studying foundational papers or production compilers in isolation.

The primary contributions of this thesis are:
\begin{enumerate}
    \item \textbf{A Didactic Implementation of the Bidirectional Algorithm:} A clear, step-by-step implementation of the core type inference algorithms from \cite{jones-practical-2007}, designed specifically to make the mechanism of the constraint-based checking using levels understandable.
    \item \textbf{A Modern, Constraint-Based Architecture:} A deliberate architectural choice to implement a GHC-style, two-phase inference engine. This separates \textbf{constraint generation} from \textbf{constraint solving}, providing a modular and pedagogically superior alternative to the eager unification model.
    \item \textbf{An Interactive Toolchain:} A complete language toolchain, featuring a parser, a typechecker, an evaluator, and a Language Server Protocol (LSP) implementation. The LSP makes the results of the type inference pipeline directly visible and interactive, transforming abstract rules into concrete feedback.
    \item \textbf{A Public Repository:} A publicly available and permissively licensed codebase \cite{deemp-arbitrary-rank-tutorial} to serve as a community resource for learning and experimentation.
\end{enumerate}

\newpage
% Section 4: Research Questions - Sharpened to reflect the specific architectural choices.
\section{Research Questions}

The development of this thesis is guided by the following research questions:
\begin{enumerate}
    \item How can the core algorithms of a bidirectional type system for arbitrary-rank polymorphism, including subsumption and deep skolemization, be implemented in a clear and modular manner for educational purposes?
    \item How can a constraint-based inference model serve as a clearer pedagogical tool for explaining type inference than the eager unification model presented in the foundational literature?
    \item How can the Language Server Protocol be leveraged to create an interactive development environment that makes the behavior and results of a language's type inference pipeline transparent and explorable?
\end{enumerate}

\section{Thesis Outline}

This thesis is structured as follows:

\begin{itemize}
    \item \textbf{\Cref{chap:LiteratureReview}} provides the theoretical foundations for this work by reviewing the evolution of polymorphic type systems. It traces the path from the Hindley-Milner compromise to the practical, bidirectional approach for arbitrary-rank types that forms the theoretical core of this thesis.
    \item \textbf{\Cref{chap:DesignAndMethodology}} details the design and implementation of the \Arralac language. It covers the full compilation pipeline, with a special focus on the constraint-based architecture of the typechecker and the Trees That Grow AST representation.
    \item \textbf{\Cref{chap:AnalysisAndDiscussion}} evaluates the resulting system against its pedagogical goals, analyzing the trade-offs of its architectural decisions and discussing the insights gained during development.
    \item \textbf{\Cref{chap:Conclusion}} summarizes the contributions of this work, revisits the central thesis in light of the results, and outlines potential directions for future research and development.
\end{itemize}