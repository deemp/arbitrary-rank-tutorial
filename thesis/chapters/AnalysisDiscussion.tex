\chapter{Analysis and Discussion}
\label{chap:AnalysisAndDiscussion}

This chapter moves beyond description to provide a critical analysis of the \texttt{Arralac} project. The analysis will proceed in three parts. First, I will evaluate how the chosen design patterns successfully achieved the project's primary objectives and what trade-offs they entailed. Second, I will offer a qualitative evaluation of the system's non-functional characteristics. Finally, I will critically examine the limitations of the current implementation and propose concrete directions for future research.

\section{Analyzing the Architectural Contributions}
\label{sec:Discussion:Objectives}

The primary goal of this thesis was to create a modern, tutorial-focused implementation of arbitrary-rank polymorphism by synthesizing foundational theory with modern compiler engineering practices.

\subsection[From Eager Unification to Constraint-Based Inference]{From Eager Unification \\ to Constraint-Based Inference}
The most significant architectural contribution of this work is its adoption of a two-phase, constraint-based type inference model, a departure from the eager unification algorithm presented in \cite{jones-practical-2007}. In their paper, the authors suggest that a "more principled alternative is to... get the inference algorithm to return a set of constraints, and solve them all together" \cite[Sec. 9.6]{jones-practical-2007}.

\texttt{Arralac} serves as a direct, working implementation of this suggestion. By separating constraint generation (the \texttt{Typechecker}) from constraint solving (the \texttt{Solver}), this thesis demonstrates that the benefits of this architecture are fundamental to building a modular and understandable system.

\paragraph{Architectural Benefits and Trade-offs}
This separation yielded two primary benefits:
\begin{enumerate}
    \item \textbf{Improved Modularity:} The logic of the typechecker and solver are cleanly decoupled. The \texttt{Typechecker}'s sole responsibility is to traverse the AST and emit a declarative set of \texttt{WantedConstraints}. The \texttt{Solver} operates on this abstract set of constraints, free from the complexities of AST traversal.
    \item \textbf{A Foundation for Better Error Reporting:} A constraint-based model enables a holistic view of type errors. By gathering all constraints before solving, a future version of the solver could analyze the full set of conflicts to produce a much richer diagnostic than an eager unifier that fails on the first error.
\end{enumerate}

The primary trade-off is the added complexity of the \texttt{WantedConstraints} data structure, which becomes the sole interface between the two largest components of the inference engine. This required careful engineering to ensure all necessary context, such as source locations and levels for scope checking, was correctly propagated.

\subsection{The Trees That Grow AST: A Necessity for Modern Tooling}
The second key architectural choice was the adoption of the \textbf{Trees That Grow} (TTG) pattern for the AST. The LSP requires annotating the AST with inferred types, and TTG provides a type-safe and elegant solution. As demonstrated in \Cref{sec:Implementation:AST}, the AST is parameterized by the compiler pass, and type families are used to "grow" the tree with annotations like \texttt{TcAnno} only after the typechecking pass. This provides a compile-time guarantee that type information is not accessed before it has been computed.

\section[A Qualitative Evaluation of System Characteristics]{A Qualitative Evaluation \\ of System Characteristics}
\label{sec:Discussion:Characteristics}
Evaluating \texttt{Arralac} against ISO 25010 \cite{iso-25010} sub-characteristics reveals the practical impact of its design.

\begin{itemize}
    \item \textbf{Modularity:} The division of the codebase into 86 modules (\Cref{table:cloc}) is a direct outcome of the pipeline architecture. The strict separation between stages creates a highly modular system. This modularity makes the codebase highly \textbf{analysable}; a student can study the \texttt{Solver} in isolation.
    \item \textbf{Analysability (Error Handling):} The system's analysability is enhanced by its error handling strategy. By defining distinct error types for each pipeline stage, each capturing a full call stack, the system provides transparent diagnostics.
    \item \textbf{Installability and Portability:} The use of Nix for dependency management and installation is critical to the project's goal of being a reproducible tutorial artifact. It guarantees that any (Linux or macOS) user can build and run the software with a single command.
\end{itemize}

\section{Limitations and Avenues for Future Work}
\label{sec:Discussion:Limitations}
A critical analysis requires acknowledging the project's limitations. These simplifications, however, suggest clear directions for future research.

\begin{enumerate}
    \item \textbf{Lack of \texttt{let}-generalization:} The most significant functional limitation is the lack of ML-style generalization for local \texttt{let}-bindings. Future work could add a scoped-solving phase at each \texttt{let}-binding, quantifying over any metavariables whose level is local to the binding's scope, as discussed in \cite{wits-type-inference-using-constraints}.

    \item \textbf{Untyped Core Language:} \texttt{Arralac} desugars its typed AST into an untyped lambda calculus. A major extension would be to design a typed Core language and extend the desugaring process to generate the necessary type abstractions and applications, making the entire pipeline type-safe.

    \item \textbf{Simplified Constraint Solver:} The current solver is basic. It halts on the first unsolvable constraint and does not implement advanced strategies, like floating constraints out of implications and promotion \footnote{See \href{https://github.com/ghc/ghc/blob/ed38c09bd89307a7d3f219e1965a0d9743d0ca73/compiler/GHC/Tc/Utils/Unify.hs\#L2589}{Note [Unification preconditions]}}. Enhancing the solver to handle these cases and to report a complete set of residual constraints would be a valuable research project.
\end{enumerate}

In conclusion, the analysis confirms that \texttt{Arralac} is not merely a \\ reimplementation, but a modernization of the ideas in its foundational literature. It successfully serves its pedagogical purpose while providing a robust foundation for future exploration.