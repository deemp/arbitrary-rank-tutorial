\chapter{Design and Methodology}
\label{chap:DesignAndMethodology}

This chapter details the architectural design and core methodologies employed in the implementation of \texttt{Arralac}, a tutorial compiler for a lambda calculus with arbitrary-rank polymorphism. The design is heavily inspired by the bidirectional type inference system described in \textit{Practical type inference for arbitrary-rank types} \cite{jones-practical-2007}, but it incorporates several modern implementation techniques and diverges in key areas to prioritize clarity and extensibility. This chapter will first outline the overall system architecture, then justify the choice of Abstract Syntax Tree (AST) representation, and finally, delve into the specifics of the constraint-based type inference engine and the methodology used to validate its correctness.

\section{System Architecture: The Compilation Pipeline}
\label{sec:Design:Pipeline}

The process of transforming a source file from plain text into an evaluated term is managed by a multi-stage pipeline, depicted in \Cref{fig:pipeline}. Each stage performs a distinct transformation on the program representation, passing its output to the next. This standard pipeline structure was chosen to promote modularity and a clear separation of concerns, which are essential for creating an understandable and maintainable tutorial compiler.

\begin{figure}[h!]
    \centering
    \begin{tikzpicture}[
            node distance=1.5cm and 2.5cm,
            auto,
            block/.style={rectangle, draw, fill=blue!20, text width=7em, text centered, rounded corners, minimum height=3em},
            arrow/.style={-Latex, thick}
        ]
        % Define the pipeline stages as nodes
        \node[block] (reader) {Reading};
        \node[block, right=of reader] (parser) {Parsing};
        \node[block, right=of parser] (renamer) {Renaming};
        \node[block, below=of renamer] (typechecker) {Typechecking};
        \node[block, left=of typechecker] (solver) {Solving};
        \node[block, left=of solver] (zonker) {Zonking};
        \node[block, below=of zonker] (core) {Core Conversion};
        \node[block, right=of core] (evaluation) {Evaluation};

        % Draw the arrows connecting the stages
        \draw[arrow] (reader) -> (parser);
        \draw[arrow] (parser) -> (renamer);
        \draw[arrow] (renamer) -> (typechecker);
        \draw[arrow] (typechecker) -> (solver);
        \draw[arrow] (solver) -> (zonker);
        \draw[arrow] (zonker) -> (core);
        \draw[arrow] (core) -> (evaluation);
    \end{tikzpicture}
    \caption{The \texttt{Arralac} Compilation Pipeline.}
    \label{fig:pipeline}
\end{figure}

The stages are as follows:
\begin{itemize}
    \item \textbf{Reading \& Parsing:} The source text is read and transformed by a BNFC-generated parser \cite{bnfc-site-2025} into an initial AST, with each node annotated with its source location.
    \item \textbf{Renaming:} This stage performs $\alpha$-conversion, assigning a unique identifier to every variable binding to resolve shadowing and prepare the program for typechecking.
    \item \textbf{Typechecking (Constraint Generation):} The renamed AST is traversed to generate a set of type constraints. This pass gathers equality and implication constraints but does not solve them.
    \item \textbf{Solving:} A separate solver pass attempts to unify the generated constraints, performing skolem escape and occurs checks.
    \item \textbf{Zonking:} After solving, a final pass substitutes all solved metavariables with their inferred types, producing a fully-typed AST.
    \item \textbf{Core Conversion \& Evaluation:} The final, zonked AST is converted into a simpler Core language for evaluation to its weak head normal form (WHNF).
\end{itemize}

\section{Abstract Syntax Tree Design}
\label{sec:Design:AST}

The choice of AST representation is a foundational design decision. My design was guided by several key technical constraints: extensibility for future language features, the ability to annotate nodes for tooling, and smooth integration with the BNFC parser generator.

\subsection{Evaluating AST Representation Strategies}
Several alternatives were considered to meet these constraints. While the default AST generated by BNFC is simple and integrates perfectly with the parser, it lacks the necessary structural flexibility for a multi-pass compiler. Annotating the tree with pass-specific information, such as inferred types, would require modifying the core data types or resorting to less type-safe mechanisms like side-tables.

% TODO TTG uses type families
Libraries like \texttt{hypertypes} \cite{hypertypes-hackage} and \texttt{compdata} \cite{compdata-hackage} offer powerful, combinator-based approaches to building extensible ASTs. However, they introduce a significant degree of type-level programming and a conceptual overhead that was deemed counterproductive for a tutorial project aimed at clarifying compiler internals, not advanced Haskell type system features.

Therefore, the \textbf{Trees That Grow (TTG)} pattern was selected \cite{trees-that-grow-2016}. It strikes an ideal balance, offering the extensibility needed for a phased compiler while remaining conceptually straightforward. It directly models the evolution of the AST through different stages, making it a powerful pedagogical tool in its own right and mirroring the architecture of GHC \cite{ghc-gitlab-2025}.


% TODO no extension field
My implementation of TTG differs slightly from GHC's for improved flexibility. Whereas GHC uses concrete types in some fields, my implementation uses type family applications for \textit{all} fields, ensuring that every part of an AST node can be customized for a given compiler pass.

\begin{figure}
    \centering
    \begin{minted}[frame=lines]{haskell}
    -- In GHC (compiler/Language/Haskell/Syntax/Expr.hs)
    data HsExpr p
      = HsVar (XVar p) (LIdP p) -- LIdP is a type synonym, not a type family
      ...
    
    type LIdP p = XRec p (IdP p)
    \end{minted}
    \caption{GHC's TTG AST Structure (Simplified)}
\end{figure}

\begin{figure}
    \centering
    \begin{minted}[frame=lines]{haskell}
    -- In Arralac (Language/Arralac/Syntax/TTG/SynTerm.hs)
    data SynTerm x
      = SynTerm'Var (XSynTerm'Var' x) (XSynTerm'Var x) -- All fields use type families
      ...
    
    -- Type families for each field, defined separately
    type family XSynTerm'Var' x
    type family XSynTerm'Var x
    \end{minted}
    \caption{Arralac's TTG AST Structure}
\end{figure}

This uniform use of type families provides a type-safe guarantee that pass-specific information, such as inferred types, is only available in the AST after that pass has successfully completed.

\section{Type Inference and Constraint Solving}
\label{sec:Design:TypeInference}

The design of the \texttt{Arralac} typechecker required a concrete implementation of several core concepts from the bidirectional system of \cite{jones-practical-2007}. This section outlines these foundational mechanisms that the methodology must realize and then describes the significant architectural divergence taken in \texttt{Arralac}.

\subsection{Foundations from \textit{Practical Type Inference}}
\label{sec:Design:Foundations}

The algorithm in \cite{jones-practical-2007} provides a practical way to handle higher-rank types by leveraging programmer annotations. Its key mechanisms are as follows:

\begin{itemize}
    \item \textbf{Type Hierarchy:} Types are stratified into monotypes (\textbf{$\tau$}), rho-types (\textbf{$\rho$}), and polytypes (\textbf{$\sigma$}) to manage polymorphism within a type context (\textbf{$\Gamma$}).
    \item \textbf{Metavariables and Unification:} Inference proceeds by creating placeholder \textbf{metavariables} for unknown types and solving equality constraints via \textbf{unification}. The system enforces a crucial \textbf{monotype invariant}: metavariables can only be unified with $\tau$-types to ensure decidability.
    \item \textbf{Bidirectional Type Checking:} The algorithm operates in an \textbf{inference mode} ($\uparrow$) to synthesize types and a \textbf{checking mode} ($\downarrow$) to verify expressions against known types. This duality avoids undecidable full inference for higher-rank types.
          % TODO when does it happen?
    \item \textbf{Subsumption and Skolemization:} The "more polymorphic than" relation is handled by \textbf{subsumption}, which is implemented via \textbf{deep skolemization}. Quantified variables are replaced with rigid "skolem" constants to check for valid instantiation and prevent scope escape.
\end{itemize}

\subsection{Architectural Divergence: A Constraint-Based Model}
\label{sec:Design:ArralacApproach}

While the theoretical underpinnings are rooted in \cite{jones-practical-2007}, the implementation of the inference mechanism diverges significantly. Instead of the \textbf{eager unification} described in the paper, where constraints are solved as they are discovered, \texttt{Arralac} adopts a \textbf{constraint-based} approach inspired by GHC. This architectural choice was made to enhance modularity and to lay a more robust foundation for high-quality error diagnostics, reflecting modern compiler practice. \Cref{tab:arch-comparison} summarizes the key differences.

\begin{table}[h!]
    \centering
    \small
    \caption{Comparison of Type Inference Architectures}
    \label{tab:arch-comparison}
    \begin{tabular}{p{0.2\textwidth} p{0.35\textwidth} p{0.35\textwidth}}
        \toprule
        \textbf{Feature}            & \textbf{\cite{jones-practical-2007} (Eager Unification)}                            & \textbf{\texttt{Arralac} (Constraint-Based)}                                                                                            \\
        \midrule
        \textbf{Unification}        & Solves constraints immediately as they are generated.                               & Gathers all constraints first; solves them in a separate, dedicated pass.                                                               \\
        \textbf{Modularity}         & Inference logic is tightly coupled with unification logic.                          & The Typechecker (generation) and Solver (unification) are fully decoupled modules.                                                      \\
        \textbf{Error Reporting}    & rrors are reported at the first point of unification failure, which can be obscure. & Already reports the location of a sub-term where a constraint originated. Has the potential for holistic error analysis by examining all conflicting constraints at once. \\
        \textbf{Let-Generalization} & Described via a global context traversal at the point of the \texttt{let}-binding.  & Not implemented, but the architecture would require a more explicit, scoped solving of constraints.                                     \\
        \bottomrule
    \end{tabular}
\end{table}

\newpage
This separation is realized through two primary constraint types:
\begin{enumerate}
    \item \textbf{Canonical Equality Constraints (\texttt{EqCt}):} Represent a required equality between a metavariable and a type.
    \item \textbf{Implication Constraints (\texttt{Implication}):} Capture the scoping of polymorphism. An implication bundles a set of skolem variables with the \textbf{wanted} \cite{wits-type-inference-using-constraints} constraints generated from checking an expression within their scope.
\end{enumerate}

% TODO can we unify with a metavariable created at a higher level?
% metavariable or a 
Furthermore, \texttt{Arralac} manages polymorphism scoping using \textbf{levels}, a technique also used in modern compilers \cite{practical-type-inference-with-levels-2025}. Every variable is assigned an integer \texttt{TcLevel} at creation. The solver then enforces the skolem escape check by a simple rule: a metavariable at level $n$ cannot be unified with a type containing any skolem variable from a level $m > n$.

\section{Validation Methodology}
\label{sec:Implementation:Methodology}

% TODO generate test cases
To validate the correctness of the implementation and ensure it meets its pedagogical goals, a methodology of targeted, feature-driven testing was employed. Rather than relying on a large, undifferentiated test suite, specific test cases were developed to exercise the core mechanisms of the arbitrary-rank type system and its implementation.

\begin{enumerate}
    \item \textbf{Correct Handling of Higher-Rank Polymorphism:} The primary validation case involved a program that requires passing a polymorphic function as an argument, similar to \texttt{Program1.arralac} (\cref{sec:Implementation:Results}). The successful typechecking and evaluation of this program served as the baseline validation that the core bidirectional algorithm, including subsumption and deep skolemization, was implemented correctly.

    \item \textbf{Skolem Escape and Level Checking:} A specific negative test case was created to ensure that a skolem variable could not escape its scope during unification. The test involved attempting to unify a metavariable with a type containing a skolem from a deeper scope. The expected outcome was a type error from the solver, which validated that the level-based checking mechanism was correctly preventing unsound unifications.

    \item \textbf{Language Server Functionality:} The interactive tooling was validated manually within Visual Studio Code. This involved checking that (a) hovering over identifiers displayed the correct, fully-zonked types as inferred by the pipeline, and (b) introducing syntactic or semantic errors (e.g., unbound variables) triggered appropriate and timely error diagnostics from the language server.
\end{enumerate}

While this methodology does not constitute a formal proof of correctness, it provides enough evidence that the key features of the system are implemented correctly and robustly, satisfying the primary objectives of the thesis.

\section{Summary of Design Choices and Limitations}
\label{sec:Design:Summary}

The design of \texttt{Arralac} makes several deliberate trade-offs to prioritize its tutorial nature and extensibility over feature completeness. The key design choices were:
\begin{itemize}
    \item An extensible \textbf{Trees That Grow} AST to support future language features and tooling annotations.
    \item A GHC-style, two-phase \textbf{constraint-based type inference} engine, which separates constraint generation from solving.
    \item \textbf{Level-based scoping} for skolem variables, providing an efficient mechanism for escape checking.
\end{itemize}

This design, however, comes with several limitations compared to a production compiler. The accompanying Technical Appendix to the paper \cite{practical-type-inference-proofs} provides proofs of soundness and completeness for the theoretical system, but this implementation does not attempt to formally prove its own correctness. The most notable limitations are:
\begin{itemize}
    \item \textbf{No \texttt{let}-generalization:} Local \texttt{let}-bindings are not generalized, a simplification suggested in \cite{vytiniotis-outsideinx-2011}.
    \item \textbf{No recursive \texttt{let}-bindings.} just like in \cite{jones-practical-2007}.
    \item \textbf{No floating-out of constraints:} The solver does not attempt to move equality constraints out of implications to enable further solving.
    \item \textbf{Untyped Core Language:} Unlike GHC and \cite{jones-practical-2007}, \texttt{Arralac} translates \texttt{SynTerm}s to a simple, untyped Core language, forgoing the powerful consistency checks that a typed intermediate language provides. This simplification was a deliberate trade-off to keep the focus of the thesis squarely on the front-end type inference algorithm.
    \item \textbf{Simplified Constraint Solving:} The solver does not rewrite Wanteds with Wanteds \footnote{See \href{https://github.com/ghc/ghc/blob/ed38c09bd89307a7d3f219e1965a0d9743d0ca73/compiler/GHC/Tc/Types/Constraint.hs\#L2415}{Note [Wanteds rewrite Wanteds]}} and only reports the first constraint that it could not solve, not all residual constraints.
\end{itemize}